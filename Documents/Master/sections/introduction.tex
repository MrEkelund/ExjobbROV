\chapter{Introduction}\label{cha:intro}
This is the master's thesis \textit{Model-based Design Development and Control of an Underwater Vehicle}.
This master's thesis was performed at Combine in Linköping.

%%%%%%%%%%%%%%%%%%%%%%%%%%%%%%%%%%%%%%%%%%%%%%%%%%%%%%%%
\section{Background}
During the recent years there have been an explosive growth in popularity and public availability of drones and unmanned vehicles (\abbrUV:s) \citep{popmechanics}. With this increased popularity, some new remotely operated underwater vehicles (\abbrROV:s) have been made available for public purchase. There have been new releases like the BlueROV from Blue Robotics \citep{bluerobotics} and the Trident from Open ROV \citep{openrov}. With open source products like the aforementioned \abbrROV:s being readily available, the subject of underwater navigation and control has become more and more relevant to hobbyists and enthusiasts.

\abbrROV:s have a large area of application and commercial \abbrROV:s are at the present time used for inspection of naval structures, seabed examination, underwater welding, ship cleaning, object location and recovery \citep{saab}, while the open source products are more oriented towards exploration. It was of special interest for us to investigate how the control systems of a open source \abbrROV solution, in this thesis the BlueROV from Blue Robotics, could be developed via model-based design and control. The possibility of autonomous operation and underwater positioning was also of interest.

Since a typical \abbrROV solution has 6 degrees of freedom (\abbrDOF) and most often isn't decoupled, it is advantageous to use a control system when executing advanced manoeuvres during exploration and missions. The controller structure originally implemented in the BlueROV platform was an open-loop controller with ad hoc decoupling. This type of control is somewhat capable during manual operation with low requirements on accuracy but might be too inexact in autonomous and more delicate operation. 

Autonomous operation places special requirements on a control system. This is due to safety and precision requirements during operation \citep[p.416-417]{safety}. To meet these needs, a model-based control strategy might be used. A model-based control needs a good model of the system. A model can be created via some base knowledge of the system and the underlying physics, via system identification or a combination of both \citet{modellbygge}. 

A typical \abbrUV uses a GPS unit to estimate its position and to improve the velocity estimates. Unfortunately GPS signals quickly lose strength in underwater environments, which in turn places extra importance in how system identification of \abbrROV platforms is performed.


%%%%%%%%%%%%%%%%%%%%%%%%%%%%%%%%%%%%%%%%%%%%%%%%%%%%%%%%
\section{Purpose}
The purpose of this master's thesis was to show how model-based design development could be used to implement a robust control system for a \abbrROV. The result this thesis will also be an input for future work regarding control of nautical vehicles. 

%%%%%%%%%%%%%%%%%%%%%%%%%%%%%%%%%%%%%%%%%%%%%%%%%%%%%%%%

\section{Goals}
The goals of this thesis were to develop a model of a \abbrROV and to use the model for developing a robust control system for the \abbrROV.

\subsubsection{Sub-goals}
To get a better overview, the goals have been divided into the following sub-goals:
\begin{itemize}
    \item Assemble the \abbrROV.
    \item Develop a framework for changing controllers in the \abbrROV.
    \item Estimate a model of the \abbrROV.
    \item Create a plant model of the \abbrROV in MATLAB/Simulink.
    \item Develop a robust model-based controller and evaluate its performance against a controller without a model.
\end{itemize}

%%%%%%%%%%%%%%%%%%%%%%%%%%%%%%%%%%%%%%%%%%%%%%%%%%%%%%%%
\section{Methodology}
At first, a theoretical study of the \abbrROV's model was performed. A literature study was also performed to gain experience of earlier studies. Then a plan for estimating the model parameters was formulated. Different approaches of system identification were tested and well thought-out before experiments were conducted. The parameter estimation was iterated several times using several methods to get well-estimated parameters. 

The \abbrROV's computer system was built on top of Robot Operating System (\abbrROS) using several different packages, a list of dependencies is available in \appref{app:dependencies}. The software was implemented with the divide and conqueror method, \textit{i.e.} the software nodes were implemented stepwise with increasing complexity. The different nodes in the system had only basal communication in the beginning and were developed to contain more complex functions, such as sensor fusion and controllers. 

Different predetermined tests were conducted to evaluate the different controllers against each other. The controllers were finely tuned before the tests and thus the most suitable controller/controllers were found.


%%%%%%%%%%%%%%%%%%%%%%%%%%%%%%%%%%%%%%%%%%%%%%%%%%%%%%%%
\section{Exclusions}
Since no absolute position measurements are available on the \abbrROV platform, this thesis will only concern parameter estimation and control in attitude and angular velocities.% Moreover, the number of different controller have been kept low to reduce implementation time and complexity.
\section{Thesis Outline}
In \Chapterref{cha:hardware}, the \abbrROV and its components are described. The \abbrROV's capabilities, physical limitations and its performance are explained and the operating system on which the \abbrROV platform is built on is briefly explained.

The necessary prerequisites for parameter estimation, namely sensor fusion and modelling, are explained in \Chapterref{cha:modelling} and \Chapterref{cha:sensor_fusion}. In \Chapterref{cha:modelling}, the coordinate systems used to model the \abbrROV are displayed and kinematic relations are presented. Furthermore, the complete \abbrROV model is presented step by step. The effects of rigid-body kinetics, hydrostatics, hydrodynamics and the \abbrROV's actuators are modelled separately and then combined to produce the complete 6-\abbrDOF model. 
Lastly in \Chapterref{cha:modelling}, ways of simplifying the model to improve identifiability and to reduce the number of parameters to be estimated are presented together with assumptions that had been made in order for the changes in the model to hold. 

In \Chapterref{cha:sensor_fusion}, an implementation of the chosen method of sensor fusion is explained, two different motion models are presented together with measurements equations for each \abbrROV sensor. Moreover, methods of outlier rejection are presented for the sensors.

In \Chapterref{cha:parameterEstimation}, the general idea and purpose of parameter estimation is explained. Two different ways of estimating parameters are presented, estimation using the prediction-error method and estimation using an \abbrEKF with an expanded state vector. Furthermore, the process of collecting and preprocessing data for parameter estimation is presented. Results using the different methods and different model structures are presented with parameter values, comparisons between simulations and validation data and fit-values. Lastly, the advantages and drawbacks of the different approaches and models are discussed and a set of parameters are chosen for controller design.\\

In \Chapterref{cha:controller}, the control problem is briefly described and different ways of state control and exact linearisation are presented. The controllers implemented on the \abbrROV are further explained and results from tests and simulations are presented. Additionally, problems encountered during tests and ideas for solving these issues are brought up and discussed.

The thesis is concluded in \Chapterref{cha:conclusions} and areas of improvement of the \abbrROV platform are brought up. 
