With the rising poplulatity of \abbrROV{}s and other \abbrUV solutions, more robust and high performance controllers have become a necessity. A model of the \abbrROV or \abbrUV can be a valuable tool during control synthesis. To use a model in design and development of controllers for an \abbrROV has been the main goal of this thesis.

In this thesis an \abbrROV from Blue Robotics has been used. It has been further developed with sensor fusion capabilities and a control systems.     
An attitude model of the \abbrROV has been estimated to aid while creating the control system. 

To model the \abbrROV, the framework of \citet{fossen2011} has been used. The model has been estimated using the prediction-error method and \abbrEKF estimation. The resulting model has been compared to validation data and describes the angular velocities well with around $70\ \%$ fit. Using the prediction-error method, it was found that the initial states of the quaternions had a large impact on the estimated parameters and the overall fit to validation data; a Kalman smoother was used to estimate the initial states. The attitude model was used to implement an attitude controller and an angular velocity controller. Furthermore, a depth controller was developed and trimmed without the use of the model.  

Performance of the controllers has been tested both in real tests and simulations. The angular velocity controller using feedback linearisation has good reference tracking. However, the attitude controller could not be stabilised using feedback linearisation. Both controllers' performance could be improved further by trimming the controllers' parameters. 

The tests of the controllers also indicate that the attitude model's parameters are erroneously estimated due to the fact that the unscaled feedback linearisation made the \abbrROV unstable. The assumption that the \abbrROV{}s center of rotation coincides with the placement of the \abbrROV{}s center of gravity is presented as a possible source of error. 