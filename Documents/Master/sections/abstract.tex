With the rising popularity of \abbrROV{}s and other \abbrUV solutions, more robust and high performance controllers have become a necessity. A model of the \abbrROV or \abbrUV can be a valuable tool during control synthesis. To use a model in design and development of controllers for an \abbrROV has been the main objective of this thesis.

In this thesis, an \abbrROV from Blue Robotics has been used. The \abbrROV has 6 thrusters placed such that the \abbrROV has 6-\abbrDOF{}s. The \abbrROV has been equipped with, an \abbrIMU , two pressure sensors and a magnetometer. The \abbrROV platform has been further developed with \abbrEKF{}-based sensor fusion and a control system. The \abbrROV can be manually controller with an Xbox controller or via a \abbrGUI. 
 
To model the \abbrROV, the framework of \citet{fossen2011} has been used. The model has been estimated using two different methods, the prediction-error method and a extended Kalman filter based method. The resulting model has been compared to validation data and describes the angular velocities well with around $70\ \%$ fit. Using the prediction-error method, it was found that the initial states of the quaternions had a large impact on the estimated parameters and the overall fit to validation data; a Kalman smoother was used to estimate the initial states. 

The estimated model was used to implement feedback linearisation, an attitude controller and an angular velocity controller. Furthermore, a depth controller was developed and trimmed without the use of the model.  
Performance of the controllers has been tested both in real tests and simulations. The angular velocity controller using feedback linearisation achieved good reference tracking. However, the attitude controller could not be stabilised by the outer control loop when using feedback linearisation. Both controllers' performance can be improved further by trimming the controllers' parameters during tests. 

The fact that the unscaled feedback linearisation made the \abbrROV unstable indicate that the attitude model is not good enough for use in feedback linearisation. To achieve stability that magnitude of the parameters had to be scaled down. The assumption that the \abbrROV{}s center of rotation coincides with the placement of the \abbrROV{}s center of gravity is presented as a possible source of error. 

In conclusion, good performance was achieved using the angular velocity. The \abbrROV is easier to control with the angular velocity controller engaged compared to controlling it using open loop. More work is needed with the model to get acceptable performance from the attitude controller. Experiments to estimate the center of rotation and the center of gravity of the \abbrROV may be helpful when further improving the model. Replacing the current \abbrESC{}s with \abbrESC{}s capable of measuring \abbrRPM may allow for better modelling of thruster dynamics. 
