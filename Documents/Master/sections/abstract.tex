With the rising popularity of \abbrROV{}s and other \abbrUV solutions, more robust and high performance controllers have become a necessity. A model of the \abbrROV or \abbrUV can be a valuable tool during control synthesis. The main objective of this thesis was to use a model in design and development of controllers for an \abbrROV.

In this thesis, an \abbrROV from Blue Robotics was used. The \abbrROV was equipped with 6 thrusters placed such that the \abbrROV was capable of moving in 6-\abbrDOF{}s. The \abbrROV was further equipped with an \abbrIMU, two pressure sensors and a magnetometer. The \abbrROV platform was further developed with \abbrEKF{}-based sensor fusion, a control system and manual control capabilities.
 
To model the \abbrROV, the framework of \citet{fossen2011} was used. The model was estimated using two different methods, the prediction-error method and an \abbrEKF-based method. Using the prediction-error method, it was found that the initial states of the quaternions had a large impact on the estimated parameters and the overall fit to validation data. A Kalman smoother was used to estimate the initial states. To circumvent the problems with the initial quaternions, an \abbrEKF was implemented to estimate the model parameters. The \abbrEKF estimator was less sensitive to deviations in the initial states and produced a better result than the prediction-error method. The resulting model was compared to validation data and described the angular velocities well with around $70\ \%$ fit.

The estimated model was used to implement feedback linearisation which was used in conjunction with an attitude controller and an angular velocity controller. Furthermore, a depth controller was developed and tuned without the use of the model.  
Performance of the controllers was tested both in real tests and simulations. The angular velocity controller using feedback linearisation achieved good reference tracking. However, the attitude controller could not stabilise the system while using feedback linearisation. Both controllers' performance could be improved further by tuning the controllers' parameters during tests. 

The fact that the feedback linearisation made the \abbrROV unstable, indicates that the attitude model is not good enough for use in feedback linearisation. To achieve stability, the magnitude of the parameters in the feedback linearisation were scaled down. The assumption that the \abbrROV{}'s center of rotation coincides with the placement of the \abbrROV{}'s center of gravity was presented as a possible source of error. 

In conclusion, good performance was achieved using the angular velocity controller. The \abbrROV was easier to control with the angular velocity controller engaged compared to controlling it in open loop. More work is needed with the model to get acceptable performance from the attitude controller. Experiments to estimate the center of rotation and the center of gravity of the \abbrROV may be helpful when further improving the model.
