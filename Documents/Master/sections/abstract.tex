In this thesis, an attitude model for a \abbrROV has been estimated. The \abbrROV used in the thesis has been the BlueROV from Blue Robotics, software has been developed to enable the usage of the \abbrROV. To get an estimate of the \abbrROV's states, an observer has been used. The observer made use of an \abbrEKF using simplified motion models.

To model the \abbrROV, the framework of \citet{fossen2011} has been used. The attitude model has been estimated using prediction-error method and extended Kalman filter estimation. The resulting model has been compared to validation data and describes the angular velocities well, around $70\ \%$ fit. Using the prediction-error method, it was found that initial states of the quaternions had a large impact on the estimated parameters and overall fit to validation data. Initial low fit in $p$ might be due to faulty assumptions of were the \abbrROV's rotational center is located. The attitude model was used to implement an attitude controller and an angular velocity controller. 

Performance of the controllers has been tested both in real tests and simulations. The angular velocity controller using exact linearisation has good reference tracking. However, the attitude controller could not be stabilised using exact linearisation. Both controllers performance could be improved by trimming the controller parameters. 
The tests of the controllers also indicates that the attitude model's parameters are erroneously estimated due to the fact that the unscaled exact linearisation made the \abbrROV unstable.