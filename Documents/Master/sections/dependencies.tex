\chapter{Dependencies and Installation}\label{app:dependencies}
The following dependencies are some of the needed packages for compiling and running the \abbrROV.
\begin{itemize}
    \item Ubuntu 14.04
    \item ROS Indigo Igloo
    \item rosserial (Indigo)
    \item rosserial\_arduino (Indigo)
    \item image\_common (Indigo) 
    \item image\_transport (Indigo)
    \item joy (Indigo)
\end{itemize}

More details of the necessary packages are listed in the installation guide.

%%%%%%%%%%%%%%%%%%%%%%%%%%%%%%%%%%%%%%%%%%%%%%%%%%%%%%%%%%%%%%%%%%%%%%%%%%%%%%%%%%%
\section{Raspberry Pi Installation}
This section summarises how to setup the Raspberry Pi. It is assumed that the Raspberry Pi is already running Ubuntu 14.04.

\subsection{\abbrROS Installation}
To install \abbrROS the \abbrROS source needs to be in the source listing run

\begin{lstlisting}[language=bash]
sudo sh -c 'echo "deb http://packages.ros.org/ros/ubuntu $(lsb_release -sc) main" > /etc/apt/sources.list.d/ros-latest.list' && sudo apt-get install ros-indigo-ros-base -y && sudo rosdep init && rosdep update && echo "source /opt/ros/indigo/setup.bash" >> ~/.bashrc && source ~/.bashrc && sudo apt-get upgrade -y && sudo ln -s /usr /opt/vc
\end{lstlisting}
to install \abbrROS and source the necessary files.

The packages needed and some other packages used in the \abbrROV is installed by
\begin{lstlisting}[language=bash]
sudo apt-get install libraspberrypi-bin libraspberrypi-dev openssh-server build-essential avahi-daemon linux-firmware python-rosinstall ros-indigo-rosserial ros-indigo-rosserial-arduino ros-indigo-image-common ros-indigo-image-transport-plugins git && sudo sh -c 'echo "start_x=1\ngpu_mem=128" >> /boot/config.txt'
\end{lstlisting}
then restart the Raspberry pi.

\subsection{Tether Setup}
For communication with the workstation the hostnames and Ethernet port needs to be setup.
Configure the host file 
\begin{lstlisting}
sudo nano /etc/hosts
\end{lstlisting}
add
\begin{lstlisting}
10.0.0.20 bluerov
10.0.0.10 workstation
\end{lstlisting}
to the host file. To use a static IP for the \abbrROV configure the interfaces file
\begin{lstlisting}
sudo nano /etc/network/interfaces
\end{lstlisting}
add 
\begin{lstlisting}
auto eth0
iface eth0 inet static
    address 10.0.0.20
    netmask 255.255.255.0
\end{lstlisting}
to the interfaces file. Reboot the computer or restart the networking services for changes to take effect.

\subsection{Installation of the \abbrROV Package}
To download and to compile the package run
\begin{lstlisting}
git clone https://github.com/MrEkelund/ExjobbROV.git && cd ExjobbROV/catkin_ws && catkin_make && echo "source ~/path-to-package/ExjobbROV/catkin_ws/devel/setup.bash" >> ~/.bashrc 
\end{lstlisting}

Setup the communication with the arduino.
\begin{lstlisting}
sudo cp ExjobbROV/catkin_ws/src/bluerov/extra/99-bluerov.rules /etc/udev/rules.d/ && sudo udevadm trigger
\end{lstlisting}

Compile the arduino code and flash it to the arduino by running
\begin{lstlisting}
cd ExjobbROV/catkin_ws && catkin_make bluerov_arduino_firmware && catkin_make bluerov_arduino_firmware-upload
\end{lstlisting}

%%%%%%%%%%%%%%%%%%%%%%%%%%%%%%%%%%%%%%%%%%%%%%%%%%%%%%%%%%%%%%%%%%%%%%%%%%%%%%%%%%%
\section{Workstation Installation}
This section summarises how to setup the workstation for operation with the \abbrROV. It's assumed that the workstation is already running Ubuntu 14.04. 

\subsection{\abbrROS Installation}
To install \abbrROS run the following command
\begin{lstlisting}[language=bash]
sudo sh -c 'echo "deb http://packages.ros.org/ros/ubuntu $(lsb_release -sc) main" > /etc/apt/sources.list.d/ros-latest.list' && sudo apt-get install ros-indigo-ros-desktop-full && sudo rosdep init && rosdep update && echo "source /opt/ros/indigo/setup.bash" >> ~/.bashrc && source ~/.bashrc
\end{lstlisting}
then run 
\begin{lstlisting}
sudo apt-get install python-rosinstall ros-indigo-rosserial ros-indigo-rosserial-arduino ros-indigo-image-common ros-indigo-image-transport-plugins ros-indigo-joy sshpass git
\end{lstlisting}
to install some necessary packages.

\subsection{Tether Setup}
For communication with the \abbrROV the hostnames and Ethernet port needs to be setup.
Configure the host file 
\begin{lstlisting}
sudo nano /etc/hosts
\end{lstlisting}
add
\begin{lstlisting}
10.0.0.20 bluerov
10.0.0.10 workstation
\end{lstlisting}
to the host file. To use a static IP for the workstation configure the interfaces file
\begin{lstlisting}
sudo nano /etc/network/interfaces
\end{lstlisting}
add 
\begin{lstlisting}
auto eth0
iface eth0 inet static
    address 10.0.0.10
    netmask 255.255.255.0
\end{lstlisting}
to the interfaces file. Reboot the computer or restart the networking services for changes to take effect. The static IP can also be set in the network configuration GUI for Ubuntu.

\subsection{Installation of the \abbrROV Package}
To download and compile the package run
\begin{lstlisting}
git clone https://github.com/MrEkelund/ExjobbROV.git && cd ExjobbROV/catkin_ws && catkin_make
\end{lstlisting}