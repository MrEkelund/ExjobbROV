\chapter{Conclusions and Future Work}\label{cha:conclusions}
In this thesis, an attitude model for a \abbrROV has been estimated. The model has then been used in controller synthesis to get satisfactory performance from the controllers. This chapter will discuss and summarise the results from \Chapterref{cha:modelling} - \Chapterref{cha:controller}. 
Lastly, ideas for future work and development will be presented.

%%%%%%%%%%%%%%%%%%%%%%%%%%%%%%%%%%%%%
\section{Conclusions}
The estimated parameters could be used for controller synthesis but problems during parameter estimation made it take longer time than expected. The problem was mainly that fit in $p$ was low in all early validation tests. It was not until late in the project that the source of the problem was identified as the effect of thruster 6. Thruster 6 is  placed on the keel of the \abbrROV and its actuation was thought to induce both $p$- and translational $y$-motion. This assumption was, as it would seem, unfortunately not entirely true. After a initial response in $p$, either the hydrostatic restoring moment from the \abbrROV's buoyancy or damping from the translational movement caused a drop in $p$ response. No progress was made in determine which of the  out \todo[inline]{Continoue here!}    
%%%%%%%%%%%%%%%%%%%%%%%%%%%%%%%%%%%%%
\section{Future Work}
To get a better model of the \abbrROV a way of estimating the \abbrROV's linear velocities could be introduced. This could be done using hydrophones, a sound-emitter and a software capable of fast calculations of cross-correlation.
With an estimate of linear velocity the complete 6-\abbrDOF model could be used which would allow for a better exact-linearisation control law. A full 6-\abbrDOF model could possibly model the effects noted in \figureref{fig:thruster6}, which could lead to increased performance for attitude controllers.

As stated in \sectionref{sec:controlDiscussion} an upgrade to the Blue\abbrESC:s might be done in order to properly model the relation between thrust and \abbrRPM instead of drawing on a look-up table.

It was also noted that magnetometers are tricky to include in sensor fusion algorithms if the readings from the magnetometer are not normed to a origin centred sphere. This since offsets in magnetometer readings would lock the estimated \psi-angle and ruining any hope of controlling \psi. 