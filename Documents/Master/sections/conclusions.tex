\chapter{Conclusions and Future Work}\label{cha:conclusions}
In this thesis, an attitude model for a \abbrROV has been estimated. The model has then been used in controller synthesis to get satisfactory performance from the controllers. This chapter will discuss and summarise the results from \Chapterref{cha:modelling} - \Chapterref{cha:controller}. 
Lastly, ideas for future work and development will be presented.

%%%%%%%%%%%%%%%%%%%%%%%%%%%%%%%%%%%%%
\section{Conclusions}
The estimated model could be used for controller synthesis but the estimation process took longer time than expected. The problem was mainly that fit in $p$ was low in all early validation tests. It was not until late in the project that the source of the problem was identified as the effect of thruster 6. Thruster 6 is  placed on the keel of the \abbrROV and its actuation was thought to induce both $p$- and translational $y$-motion. This assumption was, as it would seem, unfortunately not entirely true. After a initial response in $p$, either the hydrostatic restoring moment from the \abbrROV's buoyancy or damping from the translational movement caused a drop in $p$ response. No progress was made in determining which of the aforementioned ideas was the source of the issue, but model fit was increased by ignoring thruster 6 in angular-velocity dynamics.  

Out of the three controllers (attitude, angular velocity and depth), depth and angular velocity achieved the most satisfactory performance. These controllers could be used in conjunction with another to form a, subjectively, pleasant control experience for the operator. The rate controller, objectively, performed best but suffered from more steady state-error than desired. Control parameters that gave satisfactory results in simulations were to strong during live tests. This in conjunction with the estimated parameters being to large in the exact linearisation led to the \abbrROV being unstable during initial test. After scaling down both control- and model parameters, stability and reference tracking was achieved with the rate controller. Control parameters that stabilised the \abbrROV during live tests were tested in simulations using the estimated model with scaled-down parameter values in the exact linearisation. This produced unsatisfactory performance, strengthening the argument that though the parameters produced good results in validation, they were poorly estimated.

Stability could not be reached while using both exact linearisation and attitude control. The source of the issue was not identified, but is suspected to originate from the conversion between demanded acceleration in the global and local frame or from bad magnetometer performance. Rudimentary performance was reached using the attitude controller by bypassing the exact linearisation, further underlining the issue with model-parameter scale.
%%%%%%%%%%%%%%%%%%%%%%%%%%%%%%%%%%%%%
\section{Future Work}
To get a better model of the \abbrROV, a way of estimating the \abbrROV's linear velocities could be introduced. This could be done using hydrophones, a sound-emitter and software to estimate the \abbrROV's using time difference of arrival (\abbrTDOA) to estimated position and velocity. Though expensive, a Doppler velocity log could also be used to directly estimate the \abbrROV's velocity compared to the water.
With an estimate of linear velocity the complete 6-\abbrDOF model could be used which would allow for a better exact-linearisation control law. A full 6-\abbrDOF model could possibly model the effects noted in \figureref{fig:thruster6}, which could lead to increased performance for attitude controllers and allow the \abbrROV to hold its position.

An upgrade to the Blue\abbrESC:s might be done in order to properly model the relation between thrust and \abbrRPM instead of drawing on a look-up table. This would also allow for \abbrRPM to be controlled, which could lead to better performance during manoeuvres. 

It was also noted that magnetometer measurements were difficult to include in sensor fusion algorithms if the readings from the magnetometer were not normed to a origin centred sphere. Offsets in magnetometer readings would lock the estimated $\psi$-angle and ruining any hope of controlling $\psi$. Improvements could be done to the sensor fusion module by creating a small program that, using magnetometer samples, returns offsets and scaling values and displays the transformed results so that it can be validated to lie on a origin centred sphere.
Improvements could possibly be achieved by temporarily disabling the magnetometer and relying more on the gyroscope during fast manoeuvres and just use the magnetometer to eliminate drift. 