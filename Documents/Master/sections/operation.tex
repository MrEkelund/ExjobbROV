\chapter{Operation of the \abbrROV} \label{app:operation}
This Chapter describes normal operation of the \abbrROV, how to implement new controllers and some tips for troubleshooting.

\section{Wiring} \label{sec:wiring}
First connect the cat 6 Ethernet cable to the Raspberry Pi. Then connect the pressure sensor to the \abbrIC port on the HkPilot, see \Figureref{fig:i2c}.
\begin{figure}
\centering
\begin{tikzpicture}
    \node[anchor=south west,inner sep=0] (image) at (0,0) {\includegraphics[trim={0cm 0cm 0cm 0cm},clip,width=0.9\textwidth]{i2c}};
    \begin{scope}[x={(image.south east)},y={(image.north west)}]
        \draw[red,ultra thick,rounded corners] (0.48,0.71) rectangle (0.55,0.83);
        \draw[blue,ultra thick,rounded corners] (0.65,0.41) rectangle (0.55,0.69);
    \end{scope}
\end{tikzpicture}
\caption{The \abbrIC port where the pressure sensor ought to be connected is circled in red. Note that the HkPilot has to be soldered for enabling the red circled \abbrIC port. The HkPilot's output ports, were the \abbrESC ought to be connected, are circled in blue.}
\label{fig:i2c}
\end{figure}
Make sure that \abbrESC $i$ is connected to the HkPilot's output port $i$, see \Figureref{fig:i2c}. Connect the Raspberry Pi to the HkPilot by a \abbrUSB cable. By following the color coded cords, connect the thrusters and the \abbrESC:s. Connect the Raspberry Pi with the JST-XH to \abbrUSB converter. Finally, connect the battery with the JST-XH to \abbrUSB converter and to the \abbrESC:s.

%%%%%%%%%%%%%%%%%%%%%%%%%%%%%%%%%%%%%%%%%%%%%%%%
\section{Start up of the \abbrROV}
For starting the \abbrROV connect all cables according to \Sectionref{sec:wiring}.
\begin{itemize}
	\item Make sure that the battery is tightly fastened and fully charged. 
	\item Slide the cradle gently into the \abbrROV tube. 
	\item If needed apply some silicone grease to the O-rings of the end cap. Then slide the end cap into the \abbrROV tube.
	\item Insert the vent bolt in to the vent nut on the end cap.
	\item Insert the cat 6 cable to the workstation.
 \end{itemize} 
Navigate to the catkin\_ws folder on the workstation and run the start script in a terminal for the \abbrROV
\begin{lstlisting}
 ./startrov.sh
\end{lstlisting}
Then in a new terminal navigate to the catkin\_ws folder and run the workstation script
\begin{lstlisting}
 ./startworkstation.sh
\end{lstlisting} 

\section{Shutdown of the \abbrROV}
To shut down the \abbrROV run the following command
\begin{lstlisting}
 ./shutdownrasp.sh
\end{lstlisting}
otherwise can the micro SD card in the Raspberry Pi take damage.
%%%%%%%%%%%%%%%%%%%%%%%%%%%%%%%%%%%%%%%%%%%%%%%%
\section{Operating the \abbrROV}
The \abbrROV can be controlled via the \abbrGUI and by an xbox controller. Three different controllers are implemented in the \abbrROV.
Choose which controller that is enabled by Dynamic Reconfigure $\rightarrow$ controller $\rightarrow$ controllers and choose wanted controller. To chose if xbox or \abbrGUI is enabled do  Dynamic Reconfigure $\rightarrow$ controller $\rightarrow$ xbox and chose either \abbrGUI or xbox. The exact linearisation can be enabled and disabled by Dynamic Reconfigure $\rightarrow$ controller $\rightarrow$ lin\_active. Note that the \abbrROV has to be armed in either mode for the thruster to work.

\subsection{Xbox Mode}
In manual mode is the \abbrROV controlled via the xbox controller. Three different controllers can be used, the angular velocity controller depth controller and the open-loop controller. \Tableref{tab:xbox} summarises the xbox controllers default configuration.
\begin{table}[htbp]
  \centering
  \caption{\label{tab:xbox}%
    The xbox controllers default configuration. Note that X and Y button only has any effect when the depth controller is enabled.}
  \begin{tabular}{l  p{0.58\linewidth}}
    \toprule%
    \textbf{Button/Stick}  & \textbf{Description} \\
    \otoprule%
    Left trigger & Increase depth. \\
    Right trigger & Decrease depth.\\
    Left bumper & Roll negative.\\
    Right bumper & Roll positive.\\
    Left stick & Translational velocities.\\
    Right stick & Angular velocities.\\
    A button & Arm the \abbrROV.\\
    B button & Disarm the \abbrROV.\\
    X button & Decrease depth reference (depth controller).\\
    Y button & Increase depth reference (depth controller).\\
    \bottomrule%
  \end{tabular}
\end{table}

\subsection{\abbrGUI Mode}
In \abbrGUI mode is the \abbrROV's attitude, angular velocities and depth controlled by the \abbrGUI. However, the translational velocities are controlled from the xbox controller. To send reference signals, enable the wanted attitude or angular velocities. Then choose the wanted reference signals, check and uncheck start\_reference\_signals. 

\subsection{Logging Data}\label{sec:logging}
For logging data rqt-bag or the supplied script can be used. In rqt on the workstation start rqt-bag by Plugins $\rightarrow$ Logging $\rightarrow$ Bag. Start the recording of data by pressing the red circle. A menu of available topics is showed, select the topics that you want to record. Then give the logfile a name in the pop-up.
To stop recording data press the red circle and close Bag by Running $\rightarrow$ Close:Bag. To use the supplied logging scripts type 
\begin{lstlisting}
./test.sh
\end{lstlisting}
in a terminal window and follow the instructions.

%\subsection{Sending Test Signals}
%To send telegraph signals to the \abbrROV the logging script in \Sectionref{sec:logging} or dynamic reconfigure can be used. Dynamic reconfigure can be used from the command line or rqt. In rqt start dynamic reconfigure by Plugins $\rightarrow$ Configuration $\rightarrow$ Dynamic Reconfigure. When the dynamic reconfigure box is shown choose matlab\_controller then check telegraph\_signals. In the submenu choose the scaling and switch factor for each thruster. Then the wanted thrusters has to be enabled/disabled by checking/unchecking the option \textit{enable\_thrusteri}. To enable tests change the test signal from off $\rightarrow$ on.    

\subsection{Displaying the Continuous Plots}
Displaying plots in rqt is enabled by Plugins $\rightarrow$ Visualization $\rightarrow$ Plot. In the plot plugin, data can be plotted by choosing topics in the topic field this is shown in \Figureref{fig:contplot}. The active topics can be seen in the command line by running 
\begin{lstlisting}
rostopic list
\end{lstlisting}
\begin{figure}
\centering
\begin{tikzpicture}
    \node[anchor=south west,inner sep=0] (image) at (0,0) {\includegraphics[trim={0cm 0cm 0cm 0.8cm},clip,width=0.9\textwidth]{contplot}};
    \begin{scope}[x={(image.south east)},y={(image.north west)}]
        \draw[red,ultra thick,rounded corners] (0.35,0.91) rectangle (0.67,0.96);
    \end{scope}
\end{tikzpicture}
\caption{The topic field where the specification of what data should be plotted is circled in red. The green plus adds data to be plotted and the red minus removes data from the graph.}
\label{fig:contplot}
\end{figure}

For multi-array messages an extra \textit{\textbackslash data} field has to be added to the topic. Data is then accessed by indexing with hard brackets. 
\begin{example}
Plot the data with index 1 in the multi-array message on topics \textit{/rovio/imu/data}. Type /rovio/imu/data/data[1] in the into the topic field and press the green plus.  
\end{example}

\subsection{\abbrLED lights on the HKPilot}
There are several different \abbrLED lights on the HKPilot Mega 2.7, \Tableref{tab:ledStatus} summaries some of \abbrLED statuses
 \begin{table}[tbp]
  \centering
  \caption{\label{tab:ledStatus}%
    Some of the \abbrLED statues used by the HKPilot Mega 2.7.}
  \begin{tabular}{l p{0.7\linewidth}}
    \toprule%
    \textbf{Light}  & \textbf{Description} \\
    \otoprule%
    Red 				& The HKPilot is setting up sensors and a connection to \abbrROS. The red light won't turn off until the connection to \abbrROS has been established.\\
    \midrule
    Pulsating blue 	& Pulsating blue light indicate that the HKPilot has connection with the workstation.\\
    \midrule
    Yellow 			& Shows that the \abbrROV is disarmed. \\
    \midrule
    Red, yellow and blue & Shows that the \abbrROV is calibrating a sensor. \\
    \bottomrule%
  \end{tabular}
\end{table}

%%%%%%%%%%%%%%%%%%%%%%%%%%%%%%%%%%%%%%%%%%%%%%%%
\section{New Parameter Estimation and New Controllers}
If a new parameter estimation is done or if a new controller is synthesised follow this section.

\subsection{New Parameter Estimation}
When a new parameter estimation is done the exact linearisation has to be updated. The model parameters that ought to be updated is in controller.h. If the simulator shall be used, ought the EstimatedParams.mat file be updated.

\subsection{New Controllers}
To implement a new controller edit the controller source files controller.cpp and controller.h. 
For easy online trimming use dynamic reconfigure parameters in the controller. Modify the controller.config file so that all wanted parameters can be modified. The controller.config file that ought to be modified is located in ExjobbROV/catkin\_ws/src/controller/config/controller.config. To implement the new controller in the simulator modify the simulator.slx file located in ExjobbROV/simulink.

%%%%%%%%%%%%%%%%%%%%%%%%%%%%%%%%%%%%%%%%%%%%%%%%
\section{Known Issues and Troubleshooting}
This sections brings up some of the known issues and how to solve them. Some general debugging and troubleshooting is also brought up.

\subsection{Calibration of Sensors}
If drift or bias is noted in the sensor fusion or in the raw sensor reading it could be due to bad calibration of the sensors.
Calibration of the gyroscope is done by
\begin{lstlisting}
rostopic pub --once /rovio/gyro/calibrate_offsets std_msgs/Bool true
\end{lstlisting}
The calibration of the accelerometer is done by running
\begin{lstlisting}
rostopic pub --once /rovio/accelerometer/calibrate_offsets std_msgs/Bool true
\end{lstlisting}
and following the instructions that shows up.
Calibration of the magnetometer is done by the command
\begin{lstlisting}
rostopic pub --once /rovio/magnetometer/calibrate_offsets std_msgs/Bool true
\end{lstlisting}
and following the commands that shows up.
For setting the magnetic field in the sensor fusion run 
\begin{lstlisting}
./calibratemag.sh
\end{lstlisting}

\subsection{\abbrROS Debugging}
Debugging \abbrROS nodes can be done by listening on \abbrROS messages. This is done by 
\begin{lstlisting}
rostopic echo /example/topic
\end{lstlisting}
In the nodes several different log messages can be generated for example by \textit{ROS\_INFO} and \textit{ROS\_DEBUG}.

\subsection{Check Ethernet Connection}
It is assumed that the setup from \Appref{app:dependencies} is done. To check the Ethernet connection from the \abbrROV to the workstation or the other way around do
\begin{lstlisting}
#From the ROV to the workstation
ping -c 10 10.0.0.10 
#From the workstation to the ROV
ping -c 10 10.0.0.20
\end{lstlisting}
If the connection is good the output ought to be something like 64 bytes received from 10.0.0.10... Otherwise the connection has to be checked. Check that the Ethernet cable is connected and that both the \abbrROV and workstation is powered on. Check that both the workstation and the \abbrROV has the correct \abbrIP by the command
\begin{lstlisting}
ifconfig
\end{lstlisting}
The output at eth0 ought to contain the correct \abbrIP. Otherwise the setup from the \Appref{app:dependencies} has to revisited.

\subsection{One or Several Thrusters Are Unresponsive}\label{subsec:unresponsive}
To check for unresponsive thrusters run the following commands 
\begin{lstlisting}
rostopic pub --once /rovio/thrusters_enable std_msgs/Bool true
./thrusterTest.sh
\end{lstlisting}
The thruster test will run the thrusters at a minimal torque in incremental order. If one or several thrusters are unresponsive check that the thrusters are connected according to \Sectionref{sec:wiring}. If all thrusters are unresponsive check that the HKPilot Mega 2.7 and the Raspberry has power and connected to each other. Note that if the \abbrROV is disarmed no control signals will be sent to the thrusters.

\subsection{Checking and Changing Rotation of Thrusters}
The rotation of thrusters are important due to the moments they create. Thus are two thruster pairs counter-rotating. For checking the rotation of the thrusters the script mentioned in \Subsectionref{subsec:unresponsive} can be run. \Figureref{fig:thrusterRotation12} - \Figureref{fig:thrusterRotation6} shows how the different thrusters should rotate when supplied with a positive control signal. If one or more thrusters are rotating in the wrong direction, change the order the corresponding thruster cables are connected. 
%\begin{tikzpicture}
%    \node[anchor=south west,inner sep=0] (image) at (0,0) {\includegraphics[trim={25cm 2cm 0cm 0cm},clip,width=0.9\textwidth]{thrusterlocationfront}};
%    \begin{scope}[x={(image.south east)},y={(image.north west)}]
%		
%        
%        \coordinate (O) at (0.48,0.55);
%        \draw[red, ultra thick,->] (O) ++(-\coordinateRadius,0) -- ++(-0.4,0) node[anchor=north east]{\large\color{red}$\yPosition$};
%        \draw[red, ultra thick,->] (O) ++(0,-\coordinateRadius)-- ++(0,-0.5) node[anchor=south east]{\large\color{red}$\zPosition$};
%        \draw[red, ultra thick] (O) circle (\coordinateRadius);        
%        \draw[red, ultra thick] (O) ++(0,\coordinateRadius) node[above]{\large\color{red}$\xPosition$};
%		\draw[red, ultra thick] (O) node[circle,fill,inner sep=1pt]{};
%        
%        \draw[yellow, thick, <->] (O) ++(0.04,0) -- ++(0,-0.37)  node [midway, right]{\large\distance{z}{6}};
%        \draw[red, thick] (O) ++(0.04,-0.37) node[right]{\large Thruster 6};
%    \end{scope}
%\end{tikzpicture}

\begin{figure}
\centering
\begin{tikzpicture}
    \node[anchor=south west,inner sep=0] (image) at (0,0) {\includegraphics[trim={0cm 0cm 0cm 0cm},clip,width=0.9\textwidth]{thrusterRotation12}};
    \begin{scope}[x={(image.south east)},y={(image.north west)}]
    		\pgfmathsetmacro\coordinateRadius{0.025}
        \pgfmathsetmacro\by{\coordinateRadius*sin(45)}
        \pgfmathsetmacro\bx{\coordinateRadius*cos(45)}
        \pgfmathsetmacro\ay{\coordinateRadius*sin(\coordRot)}
        \pgfmathsetmacro\ax{\coordinateRadius*cos(\coordRot)}
        \pgfmathsetmacro\coordYRot{sin(\coordRot)}
        \pgfmathsetmacro\coordXRot{cos(\coordRot)}
        
        \coordinate (left) at (0.28,0.205);
        \coordinate (right) at (0.78,0.205);
        \draw[->,red,ultra thick,rounded corners] (left) arc (-30:30:0.2);
        \draw[->,red,ultra thick,rounded corners] (right) arc (210:150:0.2);
 
    \end{scope}
    \coordinate (left) at (1.8,1.83);
    \draw[fill=yellow] (left) circle (0.06);
	\draw[yellow, ultra thick] (left) circle (0.25);
	\draw (left) ++(0,0.2) node[above]{\color{yellow}$\zForce$};
	\coordinate (right) at (9.85,1.85);
    \draw[fill=yellow] (right) circle (0.06);
	\draw[yellow, ultra thick] (right) circle (0.25);
	\draw (right) ++(0,0.2) node[above]{\color{yellow}$\zForce$};
\end{tikzpicture}
    \caption{The arrows indicate the direction of rotation of thruster 1 and thruster 2 when supplied with a positive control signal. The resulting positive forces are indicated in yellow.}
    \label{fig:thrusterRotation12}
\end{figure}

\begin{figure}
\centering
\begin{tikzpicture}
    \node[anchor=south west,inner sep=0] (image) at (0,0) {\includegraphics[trim={0cm 0cm 0cm 4cm},clip,width=0.9\textwidth]{thrusterRotation34}};
    \begin{scope}[x={(image.south east)},y={(image.north west)}]
        \draw[->,red,ultra thick,rounded corners] (0.24,0.405) arc (-30:30:0.2);
        \draw[->,red,ultra thick,rounded corners] (0.74,0.405) arc (210:150:0.2);
        
  	  	\coordinate (left) at (0.24,0.405);
		\draw[->, yellow, ultra thick] (left) ++(-0.12,-0.06) -- ++(-0.05,-0.25) node[right]{\color{yellow}$\xForce$};
		\coordinate (right) at (0.74,0.405);
		\draw[->, yellow, ultra thick] (right) ++(0.13,-0.06) -- ++(0.08,-0.25) node[right]{\color{yellow}$\xForce$};
    \end{scope}

\end{tikzpicture}
    \caption{The arrows indicate the direction of rotation of thruster 3 and thruster 4 when supplied with a positive control signal. The resulting positive forces are indicated in yellow.}
    \label{fig:thrusterRotation34}
\end{figure}

\begin{figure}
\centering
\begin{tikzpicture}
    \node[anchor=south west,inner sep=0] (image) at (0,0) {\includegraphics[trim={0cm 0cm 0cm 14cm},clip,width=0.9\textwidth]{thrusterRotation5}};
    \begin{scope}[x={(image.south east)},y={(image.north west)}]
        \draw[->,red,ultra thick,rounded corners] (0.44,0.445) arc (210:150:0.2);
    \end{scope}
    \coordinate (left) at (5.8,3);
    \draw[fill=yellow] (left) circle (0.06);
	\draw[yellow, ultra thick] (left) circle (0.25);
	\draw (left) ++(0,0.2) node[above]{\color{yellow}$\zForce$};
\end{tikzpicture}
    \caption{The arrows indicate the direction of rotation of thruster 5 when supplied with a positive control signal. The resulting positive forces are indicated in yellow.}
    \label{fig:thrusterRotation5}
\end{figure}

\begin{figure}
\centering
\begin{tikzpicture}
    \node[anchor=south west,inner sep=0] (image) at (0,0) {\includegraphics[trim={0cm 0cm 0cm 9cm},clip,width=0.9\textwidth]{thrusterRotation6}};
    \begin{scope}[x={(image.south east)},y={(image.north west)}]
        \draw[->,red,ultra thick,rounded corners] (0.40,0.45) arc (210:150:0.2);
    \end{scope}
    \coordinate (left) at (5.45,3.10);
    \draw[fill=yellow] (left) circle (0.06);
	\draw[yellow, ultra thick] (left) circle (0.25);
	\draw (left) ++(0,0.2) node[above]{\color{yellow}$\zForce$};
\end{tikzpicture}
    \caption{The arrows indicate the direction of rotation of thruster 6 when supplied with a positive control signal. The resulting positive forces are indicated in yellow.}
    \label{fig:thrusterRotation6}
\end{figure}
