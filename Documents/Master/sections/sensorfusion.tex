\chapter{Sensor Fusion}\label{cha:sensor_fusion}
In order to properly estimate the \abbrROV's attitude in the global coordinate system the \abbrROV needs sensors to measure external effects from its environment.
Unfortunately signals from sensors do not necessarily give direct information about attitude and their measurements are to some extent noisy. Algorithms can nevertheless be used to extract and combine the information from the different sensors into a better attitude estimate. The process of combining, or fusing, the information from several measurements with or without a motion model to produce an estimate of a state is called sensor fusion. Since sensor fusion is only a prerequisite for control of the \abbrROV's attitude no results will be presented in this section. 

To be able to understand how a sensor fusion algorithm works, it is important to understand the notation. In  \Tableref{tab:notationKalman} the notation used in this Chapter is listed.
 \begin{table}[htbp]
  \centering
  \caption{\label{tab:notationKalman}%
    The notation used for describing a sensor fusion algorithm.}
    \begin{tabular}{l p{0.7\linewidth}}
    \toprule%
    \textbf{Notation} & \textbf{Description} \\
    \otoprule%
    $\boldsymbol{x}$ & State vector.\\
    $\hat{\boldsymbol{x}}$ & State vector estimate.\\
    $\boldsymbol{y}$ & Measurement vector.\\
    $\boldsymbol{u}$ & Control signal vector.\\
    $\boldsymbol{v},\boldsymbol{e}$ & Noise vectors.\\
    $_k$ & At time $k$.\\
    $_{k|m}$ & At time $k$ given information up to time $m$.\\
    $f_{\boldsymbol{x}}$ & Jacobian of $f$ with respect to states $\boldsymbol{x}$.\\
    $f_{\boldsymbol{v}}$ & Jacobian of $f$ with respects to noise $\boldsymbol{v}$.\\
    $\text{E}(\boldsymbol{x})$ & The expected value of $\boldsymbol{x}$.\\
    $\text{Cov}(\boldsymbol{x})$ & The covariance of $\boldsymbol{x}$.\\
    \bottomrule%
 \end{tabular}
\end{table}

\section{The Extended Kalman Filter}
One filter that can accomplish the task of fusing different measurements and estimating states in a non-linear dynamic system is the extended Kalman filter (\abbrEKF). The Kalman filter (\abbrKF) is a linear state-space observer, it estimates both measurable and unmeasurable states in a linear system \citep{sensorfusion}. It utilises a motion model, a model of the systems dynamics, in conjunction with measurements and linear measurement equations to provide the best possible estimate of the model's states. The Kalman filter finds the best possible linear filter for the given input $\boldsymbol{y}_{k}$ \citep{sensorfusion}. The extended Kalman filter can, unlike the regular Kalman filter, handle non-linear motion models and measurement equations. It accomplishes this by using a linearised model of the non-linear system and the measurement equations. If an \abbrEKF can provide satisfactory results depends on the rest terms from the linearisation and it is therefore dependent on the degree of non-linearity of the system and the measurement equations \citep{sensorfusion}. As rule of thumb the rest term will be small enough if the system model is close to linear and if measurements are of good quality, meaning that the signal to noise ratio (\abbrSNR) is high \citep{sensorfusion}. 

The \abbrEKF algorithm is comprised of two key steps called updates.
The time update uses the current state estimates and the user specified motion model to predict the values of the states the next time instant. The second update is called the measurement update and it uses sampled sensor data in conjunction with the user specified measurement equations to fuse the state estimates \citep{sensorfusion}.
If the measurements are independent a measurement update can be done at the arrival of each measurement without the need of a time update in between \citep[p. 170]{sensorfusion}.

The complete extended Kalman filter algorithm is summarised in \Algoref{alg:EKF}.
\begin{algorithm}
\caption{The extended Kalman filter algorithm \citep{sensorfusion}.}
\label{alg:EKF}
  The extended Kalman filter applied on a model
    \begin{align*}
    \boldsymbol{x}_{k+1} &= f(\boldsymbol{x}_{k},\boldsymbol{u}_{k}, \boldsymbol{v}_{k})\\
    \boldsymbol{y}_{k} &= h(\boldsymbol{x}_{k},\boldsymbol{u}_{k},\boldsymbol{e}_{k})
    \end{align*} is given by the following algorithm:\\
    \textbf{Initialisation:}
    \begin{align*}
    \hat{\boldsymbol{x}}_{1|0} &= \text{E}(\boldsymbol{x}_{0})\\
    \boldsymbol{P}_{1|0} &= \text{Cov}(\boldsymbol{x}_{0})\\
    \end{align*}
     
    \textbf{Measurement update:}
    \begin{align*}
    \boldsymbol{S}_{k} &= h_{\boldsymbol{x}}(\hat{\boldsymbol{x}}_{k|k-1},\boldsymbol{u}_{k}) \boldsymbol{P}_{k|k-1} h_{\boldsymbol{x}}(\hat{\boldsymbol{x}}_{k|k-1},\boldsymbol{u}_{k})^{T} + \boldsymbol{R}_{k}\\
    \boldsymbol{K}_{k} &= \boldsymbol{P}_{k|k-1} h_{\boldsymbol{x}}(\hat{\boldsymbol{x}}_{k|k-1},\boldsymbol{u}_{k})^{T} \boldsymbol{S}_{k}^{-1}\\
    \boldsymbol{\epsilon} &= \boldsymbol{y}_{k} - h(\hat{\boldsymbol{x}}_{k|k-1},\boldsymbol{u}_{k})\\
    \hat{\boldsymbol{x}}_{k|k} &= \hat{\boldsymbol{x}}_{k|k-1} + \boldsymbol{K}_{k}\boldsymbol{\epsilon}\\
    \boldsymbol{P}_{k|k} &= \boldsymbol{P}_{k|k-1} - \boldsymbol{P}_{k|k-1} h_{\boldsymbol{x}}(\hat{\boldsymbol{x}}_{k|k-1},\boldsymbol{u}_{k})^{T} \boldsymbol{S}_{k}^{-1} h_{\boldsymbol{x}}(\hat{\boldsymbol{x}}_{k|k-1},\boldsymbol{u}_{k}) \boldsymbol{P}_{k|k-1}\\
    \end{align*}
    
   \textbf{Time update:}
    \begin{align*}
    \hat{\boldsymbol{x}}_{k+1|k} &= f(\hat{\boldsymbol{x}}_{k|k},\boldsymbol{u}_{k})\\
    \boldsymbol{P}_{k+1|k} &= f_{\boldsymbol{x}}(\hat{\boldsymbol{x}}_{k|k},\boldsymbol{u}_ {k})\boldsymbol{P}_{k|k} f_{\boldsymbol{x}}(\hat{\boldsymbol{x}}_{k|k},\boldsymbol{u}_{k})^{T} + f_{\boldsymbol{v}}(\hat{\boldsymbol{x}}_{k|k},\boldsymbol{u}_{k}) \boldsymbol{Q}_{k} f_{\boldsymbol{v}}(\hat{\boldsymbol{x}}_{k|k},\boldsymbol{u}_{k})^{T}
    \end{align*}    
\end{algorithm}

\section{Motion Model}\index{Motion model}\label{sec:motion_model}
A \abbrKF uses a model of the system dynamics to improve the estimates of the model's states. It is therefore very important to choose a model that describes the system's dynamics well. In this thesis, two different motion models have been used, a model using the measured angular velocities as inputs and a more advanced model using the angular velocities as states. All models in this chapter use quaternions and thus quaternion normalisation is required as described in \Chapterref{cha:modelling}.

\index{Quaternions}
The simple model using the measured angular velocities as inputs was based on the quaternion kinematics model in \citet[p. 47]{Tornqvist}. The model was expanded with depth as an extra state which was modelled as state with constant position and was discretised using Euler forward. Using the quaternion representation of $\etaVector$ the complete simple model is 
\begin{equation}
\begin{bmatrix}
\etaVector_{k+1}\\
d_{k+1}
\end{bmatrix} 
=
 \begin{bmatrix}
 \boldsymbol{I}_{4\times4} + T_s \bar{\boldsymbol{T}}(\nuVector_k) & \boldsymbol{0}_{4\times1}\\
 \boldsymbol{0}_{1\times4} & \boldsymbol{I}_{1\times1} 
 \end{bmatrix}
 \begin{bmatrix}
 \etaVector_{k}\\
 d_k
 \end{bmatrix}
 +
  \begin{bmatrix}
  T_s \boldsymbol{T}(\etaVector_{k}) & \boldsymbol{0}_{4\times1}\\
  \boldsymbol{0}_{1\times3} & T_s
  \end{bmatrix}
  \begin{bmatrix}
  \boldsymbol{v}_{\etaVector}\\
  v_d 
  \end{bmatrix}
\end{equation}
Here $\bar{\boldsymbol{T}}(\nuVector)$ is defined as
\begin{equation}
\bar{T}(\nuVector) = \frac{1}{2}
\begin{bmatrix}
     0 &-p &-q &-r\\
     p & 0 & r &-q\\
     q &-r & 0 & p\\
     r & q &-p & 0
\end{bmatrix}
\end{equation} and $\boldsymbol{T}(\etaVector)$ is given in \eqref{eq:Tquat}.
It is important to note that this is an attitude model and thus $\etaVector$ and $\nuVector$ only contains quaternions and angular velocities respectively. Note that $\nuVector$ is not modelled as a state but is used as an input to reduce the dimension of the model as in \citet{Tornqvist}. 

A second model was implemented to improve sensor fusion performance. This model was based on the continuous rigid-body kinematics model in \citet{sensorfusion}[p. 351] and incorporated gyroscope bias estimates and $\nuVector$ as a constant position states. The continuous model in \citet{sensorfusion}[p. 351] also modelled positions, linear- accelerations and velocities. Since no position measurements except for depth-measurements are available on the \abbrROV platform, linear- accelerations and velocities were ignored and the position sate was reduced to only contain depth. This was done to reduce the dimension of the model. The noise model was also slightly modified by giving each state its own noise source. This is not true from a physical standpoint since for example, a disturbance in angular velocity will effect the angles, but it made the filter easier to trim. The entire model was discretised using Euler forward which yielded the following discrete model
\begin{equation}\label{eq:expanded_model}
\begin{bmatrix}
\etaVector_{k+1}\\
\nuVector_{k+1}\\
\boldsymbol{b}_{k+1}\\
d_{k+1}
\end{bmatrix}=
\begin{bmatrix}
\boldsymbol{I}_{4\times4} + T_s \bar{\boldsymbol{T}}(\nuVector_{k})& \boldsymbol{0}_{4\times7}\\
\boldsymbol{0}_{7\times4} & \boldsymbol{I}_{7\times7}\\
\end{bmatrix}
\begin{bmatrix}
\etaVector_{k}\\
\nuVector_{k}\\
\boldsymbol{b}_{k}\\
d_{k}
\end{bmatrix}
+\begin{bmatrix}
  \boldsymbol{v}_{\etaVector}\\
  \boldsymbol{v}_{\nuVector}\\
  \boldsymbol{v}_{\boldsymbol{b}}\\
  v_d 
\end{bmatrix}
\end{equation}

\section{Measurement Equations}\label{sec:Meas}
To fuse information from different sensors the readings of each sensor has to be related with the modelled states and noise sources which is done using measurement equations.

The \abbrROV is equipped with an \abbrIMU which contains a gyroscope. Since readings from a gyroscope might not be zero in all axes even if the gyroscope is at a rest, it is important to incorporate biases in when modelling it. Otherwise a constant reading offset would imply that the gyroscope is in motion even at rest. Modelling for biases in the gyroscope gives the following gyroscope measurement equation
\begin{equation}\label{eq:gyro}
\boldsymbol{y}_\text{Gyro}= \nuVector + \boldsymbol{b} + \boldsymbol{v}_{\text{Gyro}}
\end{equation}
 The vector $\boldsymbol{y}_\text{Gyro}$ is the reading from the \abbrIMU's gyroscope in $rad/s$. The vector $\boldsymbol{b}$ contain the gyroscope's biases in the \xPosition-, \yPosition- and \zPosition-axes, respectively.
The readings from the gyroscope were of good quality and no obvious sanity check for measurements was found no outlier rejection was performed on the gyroscope measurements. %Also note that \eqref{eq:gyro} is only used in conjunction with the expanded model \eqref{eq:expanded_model} since this models $\nuVector$ as a state. 

The \abbrIMU measures acceleration in addition to angular velocities, an accelerometer measurement update can be done with each new \abbrIMU data packet. The measurement equation for the accelerometer is
\begin{equation}
\boldsymbol{y}_{\text{Acc}} =
    \boldsymbol{R^n_b}(\boldsymbol{q})^T
    \begin{bmatrix}
    0\\
    0\\
    -g
    \end{bmatrix}
+ \boldsymbol{v}_{\text{Acc}}
\end{equation}
    where $\boldsymbol{R^n_b}(\boldsymbol{q})$ is the rotation matrix defined in \Chapterref{cha:modelling}, $g$ is the gravitational constant and $\boldsymbol{v}_{\text{Acc}}$ is measurement noise. Since the accelerometer is not perfectly centred in the \abbrROV's \abbrCG and since\abbrROV rotates, accelerates and decelerates $g$ is not the only thing that is being measured by the accelerometer. This leads to problems when trying to estimate the \abbrROV's attitude since the sensor fusion algorithm tries to use the known direction and magnitude of the Earth's gravitational pull to estimate the \abbrROV's attitude. To ensure that only the gravitational constant $g$ is used to update the \abbrROV's attitude, outlier rejection is performed. Accelerometer measurements are used if 
\begin{align}
    \abs{~||
    \boldsymbol{y}_{\text{Acc}}
||
    -g
     ~} < \epsilon_{\text{Acc}}
\end{align}
Here, $\epsilon_{\text{Acc}}$ a design parameters that tweaks how much the magnitude of the accelerometer measurement may deviate from $g$ before being considered an outlier.


The \abbrIO-unit of the \abbrROV is also equipped with an internal magnetometer which enables the \abbrROV to measure the magnetic field strength in the three local axes $x$, $y$ and $z$. This data is used as a measurement in a measurement update that was constructed in the following way:
\begin{equation}
\boldsymbol{y}_{\text{Mag}} = 
    \boldsymbol{R^n_b}(\boldsymbol{q})^T
    \begin{bmatrix}
        \sqrt{m_\text{N}^2 +m_\text{E}^2}\\
        0\\
        m_\text{D}
    \end{bmatrix}
    + \boldsymbol{v}_{\text{Mag}}
\end{equation}
where $\boldsymbol{y}_{\text{Mag}}$ is the measured magnetic field in the body-fixed coordinate system and $\boldsymbol{v}_{\text{Mag}}$ is measurements noise. The parameters $m_\text{N}$, $m_\text{E}$ and $m_\text{D}$ are the the measured magnetic field in the local coordinate system at start up of the \abbrROV. Since the strength and inclination of the Earth's magnetic field vary with location $m_\text{N}$, $m_\text{E}$ and $m_\text{D}$ are set to values that represent the magnetic field at the current location. This is done via a calibration script that sets the $m_\text{N}$, $m_\text{E}$ and $m_\text{D}$ parameters to the current reading which in turn sets the current direction as the global coordinate system's North.\\
The \abbrROV is not a a noise free environment from a electromagnetic standpoint, currents in the \abbrROV's electronics may induce magnetic fields which will distort the sensor readings of the magnetometer. If such noisy measurements are used the sensor fusion will not perform well when estimating the \abbrROV's attitude. To ensure that only measurements of in condition are used an outlier rejection criteria was implemented. Magnetometer measurements are used if 
\begin{equation}
        \abs{~||
\boldsymbol{y}_{\text{Mag}}||
    -
    ||
    \begin{bmatrix}
        m_\text{N}\\
        m_\text{E}\\
        m_\text{D}
    \end{bmatrix}||
     ~} < \epsilon_{\text{Mag}}
\end{equation}
holds. Here $\epsilon_{\text{Mag}}$ a design parameters that tweaks how much the magnitude of the magnetometer measurements may deviate from the magnitude of the Earth's magnetic field before being rejected.

The \abbrROV is also equipped with a pressure sensor. The sensor is placed in the rear or stern of the \abbrROV which in turn means that the attitude of the \abbrROV needs to be taken in to account when estimating the depth.
Taking this into consideration yields the following measurement equation for the pressure sensor
\begin{equation}
 \boldsymbol{y}_{\text{Pre}}=  \rho g \left(d + \begin{bmatrix}
    0 & 0 & 1
\end{bmatrix} \boldsymbol{R}^n_b(\boldsymbol{q}) 
\begin{bmatrix}
x_{\text{offset}}\\
0\\
0
\end{bmatrix}\right)
    + \boldsymbol{v}_{\text{Pre}}
\end{equation}
Here $\boldsymbol{y}_{\text{Pre}}$ is the measured water pressure in pascal, $\rho$ is the density of the water, $d$ is the current depth in meters and $\boldsymbol{v}_{\text{Pre}}$ is measurement noise. The parameter $x_{\text{offset}}$ is the pressure sensor's offset from the \abbrROV's \abbrCO in the $\xPosition$-direction. In this case, $x_{\textrm{offset}}$ is a negative number. A basic form of outlier rejection is implemented in the pressure sensor measurement update. The update is performed if
\begin{equation}
    p \geq 0
\end{equation}
since a reading lower than zero would imply that the \abbrROV is above the water surface.